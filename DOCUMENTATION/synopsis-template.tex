% =====================================================================
% ====                                                             ====
% ==== Synopsis template, revised from diploma thesis short paper
% ====                                                             ====
% ==== Henrik B�rbak Christensen Winter 2004-2005
% ====                                                             ====
% =====================================================================


\documentclass[a4paper]{article}

\usepackage[latin1]{inputenc}
\usepackage{palatino}
\usepackage{color}

%% a point to check
\definecolor{checkcolor}{rgb}{0.75, 0.75, 0.75}
\newsavebox{\definitionbox}
\newenvironment{checkit}{%
\begin{lrbox}{\definitionbox}
\begin{minipage}[t]{0.85\textwidth}%
}%
{\end{minipage}\end{lrbox}%
\begin{center}{\colorbox{checkcolor}{\usebox{\definitionbox}}}%
\end{center}}


\title{Synopsis Template}

\author{Henrik B�rbak Christensen}

\date{August 2008}


\begin{document}

\maketitle

\sloppy

\begin{abstract}
This is a template (proposal) for your synopsis. This synopsis
outlines the key contents of a synopsis, and as a result presents
itself as a template for a synopsis.
\end{abstract}

\section{Motivation}
This synopsis is organized in the same manner as we expect your
synopsis to be. That is, use the headers as a template for your
synopsis.

The motivation for your work should be communicated as one of the
first things. Why is this interesting to you; why is it interesting
for third party (university, your company, fellow students, etc.)? 

Be sure also to include some background for the project that allows us
to understand the context.

A synopsis should be from two to four pages in length. As a synopsis
in this context is a shortened form or summary of the full work you
gradually transform your synopsis into the final report.

\section{Hypothesis/Problem statement}

Describe the problem that your project work is focusing on. Try hard
to come up with a 5-10 line hypothesis that you are capable of confirm
or falsify. It is no problem to state a problem in half a page, and
no-one will notice that it is ill-defined (not even yourself!)
However, forcing yourself to write it in five lines only, requires lot
of precision that will force you to define your project much more
accurately.

You may also express it somewhat broader as a problem statement (still
short!), but ensure that it is in a form where you can
argue/demonstrate that you have analyzed and evaluated the problem.

A fine ``problem'' in a teaching context is also to apply theory in
practice and learn about its advantages and short-comings.

An important part of this section is also assumptions and
delimitations. What assumptions do you have about the project, the
process or the environment? What subproblems do you not address or
will not address? Almost any project is capable of consuming an
infinite amount of work and you do not have unlimited time and
resources as hand---therefore it is important to explicitly delimit
the problem and state what you intend to look into and what not.

A final thing is characterization: ensure that all the concepts you
use is well-defined or else give a definition. We know what 'object'
means in the usual sense, but a term like 'component' still has many
different meanings! Which one do you use? Also if you use company
specific concepts, be sure to define them as precisely as possible.

Use typography to make definitions and problem statements stand out in
the text. It is more difficult to overview a text where important
points are written inside large chunks of less import text than it is
to find them as

\begin{checkit}
Use typography to make important points like definitions, results,
problem statements, and other text that needs to be consulted often,
stand out in the text.
\end{checkit}

\section{Method}

How do you intend to analyse the problem? What theory, books, and
papers have you read that help you to analyse, discuss, and work with
the problem?


A short summary of how you are going to confirm/falsify the
hypothesis: what prototypes do you expect to build, how will you
evaluate and measure them, what techniques and tools are you going to
use, which people will you interview, how will you document processes
and products, how will you record you progress, how will you analyze
your work, how will each phase contribute to validating the
hypothesis?

\section{(Expected) Analyses and Results}

This the main body of your report where you outline what you have
done, how it contributes to analysing the problem, the results of your
work, argue why your results are correct, relate them to theory, and
document what has been achieved.

Remember that designs are often best described and supported by
diagrams and central algorithms are best expressed in small fragments
of real or pseudo code. Text like ``the server calls the 'update'
method which next calls the 'IAmBored' method in the \ldots'' are
terrible and next to impossible to read.

Avoid narrative writing styles like ``then we did X but it did not work, so
we tried Y and it worked better''. Rather, use a (problem, analysis,
solution) format: ``Problem: (describe short and precisely), Analysis:
(describe a set of problem solutions), Solution: (describe and argue
why a given solution was chosen).''

Remember that the primary objective with your report is to demonstrate
that you master the theory introduced in the course and your
analytical skills to ``think clever thoughts'' and related and discuss
your work. It is not to program a polished product ready for shipment.


\section{Related work}

[This section may be put in front of the hypothesis section or
  integrated into the method section, if it make the flow of text more
  natural.]

In this section you outline what literature and other work your
project build upon: papers, books, links to webpages, tutorials,
etc. All references should be resolved in the reference section, that
is do not use footnotes, or put the reference directly in the
text. For an example, look how references are cited in Bardram et
al.\cite{bardram}.

It is good to address how your work extends, use, or build upon the
cited work. 

\section{Conclusion}

Your synopsis should clearly state: abstract, motivation, hypothesis,
method, and (expected) analyses and results.


% ============================================================================
% === REFERENCES =============================================================
% ============================================================================

\bibliographystyle{abbrv}

\bibliography{bib}



\end{document}
